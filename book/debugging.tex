\section{What is debugging?}
\index{debugging}
\index{bug}

\begin{quote}
    The greatest pleasure in writing is rewriting. My early drafts are always wretched. At first a general verb like ``move'' is qulified by the adverb ``quickly.'' After sixty tries I come up with a particular, possibly witty verb and drop the adverb. Originally I wrote ``poetry suddenly left me,'' which after twleve drafts became ``poetry abandoned me''---with another sentence to avoid self-pity. When my doctor told me I had diabetes, I was incredulous. I said, ``You mean I am pre-diabetic.'' Writing in this book, I changed a verb to mock my silly presumption. ```You mean I am pre-diabetic,' I explanined.'' (Donald B. Hall, \em{Essays AFter Eighty,} p. 13)
\end{quote}

Programming is error-prone.  For whimsical reasons, programming errors
are called {\bf bugs} and the process of tracking them down is called
{\bf debugging}.
\index{debugging}
\index{bug}

Three kinds of errors can occur in a program: syntax errors, runtime
errors, and semantic errors. It is useful
to distinguish between them in order to track them down more quickly.

\subsection{Syntax errors}
\index{syntax error}
\index{error!syntax}
\index{error message}

Python can only execute a program if the syntax is
correct; otherwise, the interpreter displays an error message.
{\bf Syntax} refers to the structure of a program and the rules about
that structure.
For example, parentheses have to come in matching pairs, so
{\tt (1 + 2)} is legal, but {\tt 8)} is a {\bf syntax error}.
\index{syntax}
\index{parentheses!matching}
\index{syntax}
\index{cummings, e. e.}

In English, readers can tolerate most syntax errors, which is why we
can read the poetry of e. e. cummings without spewing error messages.
Python is not so forgiving.  If there is a single syntax error
anywhere in your program, Python will display an error message and quit,
and you will not be able to run your program. During the first few
weeks of your programming career, you will probably spend a lot of
time tracking down syntax errors.  As you gain experience, you will
make fewer errors and find them faster.

\subsection{Runtime errors}
\label{runtime}

The second type of error is a runtime error, so called because the
error does not appear until after the program has started running.
These errors are also called {\bf exceptions} because they usually
indicate that something exceptional (and bad) has happened.
\index{runtime error}
\index{error!runtime}
\index{exception}
\index{safe language}
\index{language!safe}

Runtime errors are rare in the simple programs you will see in the
first few chapters, so it might be a while before you encounter one.


\subsection{Semantic errors}
\index{semantics}
\index{semantic error}
\index{error!semantic}
\index{error message}

The third type of error is the {\bf semantic error}.  If there is a
semantic error in your program, it will run successfully in the sense
that the computer will not generate any error messages, but it will
not do the right thing.  It will do something else.  Specifically, it
will do what you told it to do.

The problem is that the program you wrote is not the program you
wanted to write.  The meaning of the program (its semantics) is wrong.
Identifying semantic errors can be tricky because it requires you to work
backward by looking at the output of the program and trying to figure
out what it is doing.

\subsection{Experimental debugging}

One of the most important skills you will acquire is debugging.
Although it can be frustrating, debugging is one of the most
intellectually rich, challenging, and interesting parts of
programming.
\index{experimental debugging}
\index{debugging!experimental}

In some ways, debugging is like detective work.  You are confronted
with clues, and you have to infer the processes and events that led
to the results you see.

Debugging is also like an experimental science.  Once you have an idea
about what is going wrong, you modify your program and try again.  If
your hypothesis was correct, then you can predict the result of the
modification, and you take a step closer to a working program.  If
your hypothesis was wrong, you have to come up with a new one.  As
Sherlock Holmes pointed out, ``When you have eliminated the
impossible, whatever remains, however improbable, must be the truth.''
(A. Conan Doyle, {\em The Sign of Four})
\index{Holmes, Sherlock}
\index{Doyle, Arthur Conan}

For some people, programming and debugging are the same thing.  That
is, programming is the process of gradually debugging a program until
it does what you want.  The idea is that you should start with a
program that does {\em something} and make small modifications,
debugging them as you go, so that you always have a working program.

For example, Linux is an operating system that contains thousands of
lines of code, but it started out as a simple program Linus Torvalds
used to explore the Intel 80386 chip.  According to Larry Greenfield,
``One of Linus's earlier projects was a program that would switch
between printing AAAA and BBBB.  This later evolved to Linux.''
({\em The Linux Users' Guide} Beta Version 1).
\index{Linux}

Later chapters will make more suggestions about debugging and other
programming practices.
